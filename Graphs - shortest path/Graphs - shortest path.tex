% Graphs - Shortest path
% Shortest  path from a node to all nodes

% Algorithms
	% Dijkstra
	% Bellman-Ford
	% Others
	
\documentclass[runningheads]{llncs}

% Packages
\usepackage{graphicx}

% Command renewals
%\renewcommand\UrlFont{\color{blue}\rmfamily}
\renewcommand{\refname}{Referin\c{t}e}

% Information
\title{Proiect Analiza Algoritmilor}
\author{Chelcea Claudiu-Marian\orcidID{322CA}}
\institute{Universitatea Politehnica din Bucure\c{s}ti\\Facultatea de Automatic\u{a} \c{s}i Calculatoare\\ \email{claudiuchelcea01@gmail.com}}
%
%
%
%
%
\begin{document}
\maketitle             

\begin{abstract}
\^\\{I}n aceast\u{a} lucrare vom analiza, compara \c{s}i discuta principalii algoritmi ce abordeaz\u{a} tematica lucr\u{a}rii mele \c{s}i anume: Drumuri minime in graf - Costul minim de la un nod la toate celelalte. \\

Fiecare algoritm abordat va avea propria explica\c{t}ie \c{s}i mod de prezentare unic, dar, pentru fiecare dintre acestea, se va \^{i}ncerca urmarea unei structuri \c{s}i anume: \\

\begin{itemize}
	\item Prezentare
	\item Avantaje \c{s}i dezavantaje
	\item Complexit\u{a}\c{t}i
	\item Alte detalii
\end{itemize}

\keywords{Dijkstra \and Bellman-Ford \and Graphs \and Distan\c{t}e minime}
\end{abstract}
%
%
%
%
%
\section{Introducere}
\subsection{Descrierea problemei rezolvate}
\hspace{6pt}Grafurile nu au nevoie de o introducere. Teoria grafurilor s-a dezvoltat \^{i}mpreun\u{a} cu algebra \c{s}i au multiple aplicaţii practice, fiind strâns legate de multe ramuri ale matematicii, c\^{a}t \c{s}i ale informaticii. 
\paragraph{} Printre utilitit\u{a}\c{t}ile lor se afl\u{a}, \^{i}n special, modelarea de situa\c{t}ii din via\c{a} real\u{a}: conexiuni, re\c{t}ele de calculatoare, algoritmi de c\u{a}utare (Page Rank), h\u{a}r\c{t}i rutiere, etc... 

\paragraph{} Lucrarea curent\u{a} urm\u{a}re\c{s}te analiza algoritmilor principali folosi\c{t}i \^{i}n determinarea drumurilor minime \^{i}ntre noduri, acest lucru av\^{a}nd multe utilita\c{t}i practice:

\subsection{Aplica\c{t}ie practic\u{a} la problem\u{a}}
Printre cele mai importante aplica\c{t}ii practice rezolvate folosind drumurile minime reg\u{a}sim:
\begin{itemize}
	\item Găsirea drumului minim dintre doua locații (Google Maps, GPS etc.)
	\item Rutare in cadrul unei rețele (telefonice, de calculatoare etc.)
	\item G\u{a}sirea de sugestii (de ex. de prietenie) pe re\c{t}elele sociale
	\item Robo\c{t}i inteligen\c{t}i
\end{itemize}

\subsection{Specificarea solu\c{t}iilor alese}
\paragraph{}Pentru a rezolva problema drumurilor minime, voi aborda algoritmii Bellman-Ford \c{s}i Dijkstra. Ei vor fi detalia\c{t}i urm\u{a}rind aceea\c{s}i structur\u{a} \c{s}i vor fi testa\c{t}i, pe c\^{a}t posibil, cu seturi de date similare.
\paragraph{}Algoritmul Dijkstra se bazeaz\u{a} pe etichete aflate pe ramuri ce reprezint\u{a} distan\c{t}a dintre doua noduri. Acesta are o complexitate de O($V^2$),  unde V reprezint\u{a} num\u{a}rul de noduri, complexitate care este redus\u{a} la O($V + E * \log{V}$), unde E reprezint\u{a} numarul de muchii, atunci c\^{a}nd folosim o coad\u{a} de priorita\c{t}i.

\paragraph{}Algoritmul Bellman Ford poate fi folosit doar atunci c\u{a}nd nu exist\u{a} niciun ciclu \^{i}n graf. Acesta func\c{t}ioneaz\u{a} dup\u{a} principiul de actualizare constant\u{a} a distan\c{t}ei dintre noduri \^{i}n timp ce sunt parcurse, ca \^{i}n final s\u{a} se ating\u{a} solu\c{t}ia optim\u{a}. Complexitatea acestui algoritm este O($V*E$), unde V reprezint\u{a} num\u{a}rul de noduri, iar E num\u{a}rul de muchii.

\subsection{Criteriile de evaluare pentru solu\c{t}ia propus\u{a}}
\hspace{6pt}\^{I}n evaluarea algoritmilor voi folosi \^{i}n prim\u{a} faz\u{a} seturi de teste din surse externe, datorit\u{a} faptului c\u{a} acestea ofera \c{s}i raspunsul, astfel put\^{a}nd confirma at\^{a}t corectitudinea algoritmului, c\^{a}t \c{s}i compara cu timpii de rulare oferi\c{t}i de aceste surse.
\paragraph{}De asemenea, \^{i}mi voi crea propriile seturi de date \^{i}n \^{i}ncercarea de a exploata slabiciuni ale algoritmilor.
\paragraph{}Voi folosi seturi variate de date \c{s}i cantita\c{t}i diferite de date, iar, \^{i}n final, voi face medii din r\u{a}spunsurile pe care le-am ob\c{t}inut, at\^{a}t pe PC-ul meu, dar \c{s}i pe alte PC-uri \c{s}i compilatoare online.





%\begin{table}
%\caption{Table captions should be placed above the
%tables.}\label{tab1}
%\begin{tabular}{|l|l|l|}
%\hline
%Heading level &  Example & Font size and style\\
%\hline
%Title (centered) &  {\Large\bfseries Lecture Notes} & 14 point, bold\\
%1st-level heading &  {\large\bfseries 1 Introduction} & 12 point, bold\\
%2nd-level heading & {\bfseries 2.1 Printing Area} & 10 point, bold\\
%3rd-level heading & {\bfseries Run-in Heading in Bold.} Text follows & 10 point, bold\\
%4th-level heading & {\itshape Lowest Level Heading.} Text follows & 10 point, italic\\
%\hline
%\end{tabular}
%\end{table}
%
%
%\noindent Displayed equations are centered and set on a separate
%line.
%\begin{equation}
%x + y = z
%\end{equation}
%Please try to avoid rasterized images for line-art diagrams and
%schemas. Whenever possible, use vector graphics instead (see
%Fig.~\ref{fig1}).
%
%\begin{figure}
%% \includegraphics[width=\textwidth]{fig1.eps}
%\caption{A figure caption is always placed below the illustration.
%Please note that short captions are centered, while long ones are
%justified by the macro package automatically.} \label{fig1}
%\end{figure}
%
%\begin{theorem}
%This is a sample theorem. The run-in heading is set in bold, while
%the following text appears in italics. Definitions, lemmas,
%propositions, and corollaries are styled the same way.
%\end{theorem}
%%
%% the environments 'definition', 'lemma', 'proposition', 'corollary',
%% 'remark', and 'example' are defined in the LLNCS documentclass as well.
%%
%\begin{proof}
%Proofs, examples, and remarks have the initial word in italics,
%while the following text appears in normal font.
%\end{proof}
%For citations of references, we prefer the use of square brackets
%and consecutive numbers. Citations using labels or the author/year
%convention are also acceptable. The following bibliography provides
%a sample reference list with entries for journal
%articles~\cite{ref_article1}, an LNCS chapter~\cite{ref_lncs1}, a
%book~\cite{ref_book1}, proceedings without editors~\cite{ref_proc1},
%and a homepage~\cite{ref_url1}. Multiple citations are grouped
%\cite{ref_article1,ref_lncs1,ref_book1},
%\cite{ref_article1,ref_book1,ref_proc1,ref_url1}.
%%
%% ---- Bibliography ----
%%
%% BibTeX users should specify bibliography style 'splncs04'.
%% References will then be sorted and formatted in the correct style.
%%
%% \bibliographystyle{splncs04}
%% \bibliography{mybibliography}
%%

\begin{thebibliography}{8}
\bibitem{ref_article1}
OCW: https://ocw.cs.pub.ro/courses/pa/laboratoare/laborator-09

\bibitem{ref_article1}
OCW: https://ocw.cs.pub.ro/courses/pa/laboratoare/laborator-07

\bibitem{ref_article1}
RASFOIESC: https://www.rasfoiesc.com/educatie/matematica/Drumuri-minime-in-grafuri24.php

\bibitem{ref_article1}
\begin{verbatim}
WIKIPEDIA: https://en.wikipedia.org/wiki/Shortest_path_problem
\end{verbatim}

\bibitem{ref_article1}
\begin{verbatim}
WIKIPEDIA: https://en.wikipedia.org/wiki/Dijkstra%27s_algorithm
\end{verbatim}

\bibitem{ref_article1}
GEEKS FOR GEEKS: https://www.geeksforgeeks.org/dijkstras-shortest-path-algorithm-greedy-algo-7/

\bibitem{ref_article1}
GEEKS FOR GEEKS: https://www.geeksforgeeks.org/shortest-path-in-a-directed-graph-by-dijkstras-algorithm/

\bibitem{ref_article1}
HACKEREARTH: https://www.hackerearth.com/practice/algorithms/graphs/shortest-path-algorithms/tutorial/
\end{thebibliography}
%
%
%
%
%
\end{document}
